% % set pdf figure transparency problems
% \pdfpageattr {/Group << /S /Transparency /I true /CS /DeviceRGB>>}

\usepackage[usenames,dvipsnames,svgnames,table]{xcolor}
\usepackage{amsmath, amsthm, amssymb, bbm, bm}
\usepackage{enumerate}
\usepackage{graphicx}
\usepackage{float}      % figure inside minipage,
\usepackage{wrapfig, subcaption}
\usepackage[margin=0cm]{caption}
\usepackage[titletoc, toc]{appendix}
\usepackage{microtype}
\usepackage{blindtext}
\usepackage{tabularx}

\usepackage[boxruled, vlined, linesnumbered]{algorithm2e}
\SetAlFnt{\small}
\SetAlCapFnt{\small}
\SetAlCapNameFnt{\small}
\usepackage{algorithmic}
\algsetup{linenosize=\tiny}

\let\oldnl\nl% Store \nl in \oldnl
\newcommand{\nonl}{\renewcommand{\nl}{\let\nl\oldnl}}% Remove line number for one line

\usepackage[bookmarks=false]{hyperref}
%\usepackage[hyperpageref]{backref}
\hypersetup{
colorlinks=true, linkcolor=Sepia, citecolor=Sepia, filecolor=magenta, urlcolor=black,
}
\usepackage{url, bookmark}

\renewcommand{\qed}{\hfill \mbox{\raggedright \rule{0.1in}{0.1in}}}

% math macros
\newcommand{\reals}{\mathbb{R}}
\newcommand{\naturals}{\mathbb{N}}
\newcommand{\complex}{\mathbb{C}}

\newcommand{\abs}[1]{\ensuremath \left| #1 \right|}
\newcommand{\norm}[1]{\ensuremath \lVert#1\rVert}
\newcommand{\given}{\, \vert \,}
\providecommand{\cal}[1]{\ensuremath \mathcal{#1}}
\newcommand{\ag}[1]{\ensuremath \left\langle#1\right\rangle}
\providecommand{\OO}{\mathcal{O}}
\newcommand{\indep}{\protect\mathpalette{\protect\independenT}{\perp}}
\def\independenT#1#2{\mathrel{\rlap{$#1#2$}\mkern2mu{#1#2}}}
\newcommand{\trace}{\trm{tr}}
\newcommand{\kron}{\otimes}
\DeclareMathOperator*{\argmin}{argmin}
\DeclareMathOperator*{\argmax}{argmax}

% shortcuts
\newcommand{\aeq}[1]{\begin{align} #1 \end{align}}
\newcommand{\aeqs}[1]{\begin{align*} #1 \end{align*}}
\newcommand{\beq}[1]{\begin{equation}#1\end{equation}}
\newcommand{\beqs}[1]{\begin{equation*}#1\end{equation*}}

\newcommand{\trm}[1]{\mathrm{#1}}
\newcommand{\enum}[2][(a)]{\begin{enumerate}[#1]{#2}\end{enumerate}}
\newcommand{\clist}[1]{\begin{itemize}\setlength{\itemsep}{0pt}
\setlength{\parsep}{0pt}
{#1}\end{itemize}}
\newcommand{\ilist}[1]{\begin{itemize}{#1}\end{itemize}}
\newcommand{\bmat}[1]{\begin{bmatrix}#1\end{bmatrix}}
\newcommand{\mpage}[2]{\begin{center}
\begin{minipage}{#1}#2\end{minipage}\end{center}}
\newcommand{\la}{\ \leftarrow\ }
\newcommand{\ra}{\rightarrow}
\providecommand\f[2]{\ensuremath \frac{#1}{#2}}
\providecommand\rbrac[1]{\ensuremath \left(#1\right)}
\providecommand\sqbrac[1]{\ensuremath \left[#1\right]}
\providecommand\cbrac[1]{\ensuremath \left\{#1\right\}}

\newcommand{\deriv}[1]{\f{d}{d #1}\ }
\newcommand{\pderiv}[1]{\f{\partial}{\partial #1}\ }

\newtheorem{theorem}{Theorem}
\newtheorem{proposition}[theorem]{Proposition}
\newtheorem{lemma}[theorem]{Lemma}
\newtheorem{corollary}[theorem]{Corollary}
\newtheorem{problem}[theorem]{Problem}

\theoremstyle{definition}
\newtheorem{definition}[theorem]{Definition}
\newtheorem{example}[theorem]{Example}
\newtheorem{note}[theorem]{Note}
\newtheorem{remark}[theorem]{Remark}
\newtheorem{assumption}[theorem]{Assumption}

\renewcommand{\P}{\trm{P}}
\newcommand{\E}{\mathbb{E}}
\newcommand{\corr}{\textrm{cross-corr}\ }
\providecommand{\ones}{\mathbbm{1}}
\providecommand{\ind}{{\bf 1}}
\providecommand{\nnot}[1]{\overline{#1}}
\providecommand{\oor}{\vee}
\providecommand{\aand}{\wedge}
\renewcommand{\implies}{\Rightarrow}
\newcommand{\convp}{\overset{P}{\to}}
\newcommand{\convd}{\overset{\DD}{\to}}

\newcommand{\s}{\sigma}
\newcommand{\w}{\omega}
\renewcommand{\r}{\rho}
\renewcommand{\t}{\tau}
\renewcommand{\th}{\theta}
\renewcommand{\a}{\alpha}
\newcommand{\p}{\phi}
\newcommand{\e}{\epsilon}
\renewcommand{\b}{\beta}
\newcommand{\g}{\gamma}
\renewcommand{\d}{\delta}
\newcommand{\D}{\Delta}
\newcommand{\z}{\zeta}
\renewcommand{\L}{\Lambda}
\renewcommand{\l}{\lambda}
\newcommand{\G}{\Gamma}
\renewcommand{\S}{\Sigma}
\newcommand{\Th}{\Theta}

\def \XX {\mathcal{X}}
\def \LL {\mathcal{L}}
\def \DD {\mathcal{D}}
\def \FF {\mathcal{F}}
\def \OO {\mathcal{O}}
\def \tOO {\widetilde{\mathcal{O}}}
\def \BB {\mathcal{B}}
\def \PP {\mathcal{P}}

\newcommand{\bR}{\mathbb{R}}
\newcommand{\bN}{\mathbb{N}}
\newcommand{\bZ}{\mathbb{Z}}

\def \rmF {\trm{F}}

\newcommand{\var}{\trm{var}}
\newcommand{\stddev}{\trm{stddev}}
\newcommand{\covar}{\trm{covar}}

\newcommand{\Poissondist}{\trm{Poisson}}
\newcommand{\Berdist}{\trm{Ber}}
\newcommand{\Geomdist}{\trm{Geom}}
\newcommand{\Betadist}{\trm{Beta}}
\newcommand{\Bindist}{\trm{Bin}}
\newcommand{\Gammadist}{\trm{Gamma}}
\newcommand{\InvGammadist}{\trm{InvGamma}}
\newcommand{\Expdist}{\trm{Exp}}

\newcommand{\mozo}{\{-1, 0, 1\}}
\newcommand{\moo}{\{-1, 1\}}
\newcommand{\mooc}{[-1, 1]}
\newcommand{\zoo}{\{0, 1\}}

\newcommand{\data}{\trm{input}}
\newcommand{\convolution}{\trm{conv}}
\newcommand{\maxpool}{\textrm{max-pool}}
\newcommand{\meanpool}{\trm{meanpool}}
\newcommand{\relu}{\trm{relu}}
\newcommand{\dropout}{\trm{dropout}}
\newcommand{\linear}{\trm{linear}}
\newcommand{\softmax}{\trm{softmax}}
\newcommand{\batchnorm}{\trm{batchnorm}}
\newcommand{\block}{\trm{block}}